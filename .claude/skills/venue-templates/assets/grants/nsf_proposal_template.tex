% NSF Research Proposal Template
% For NSF Standard Grant Proposals
% Last updated: 2024
% Based on NSF PAPPG (Proposal & Award Policies & Procedures Guide)

\documentclass[11pt,letterpaper]{article}

% Required formatting
\usepackage[margin=1in]{geometry}  % 1 inch margins required
\usepackage{times}  % Times Roman font (11pt minimum)
\usepackage{graphicx}
\usepackage{amsmath}
\usepackage{amssymb}
\usepackage{cite}
\usepackage{hyperref}

% Single spacing (NSF allows single spacing)
\usepackage{setspace}
\singlespacing

% Page numbers
\usepackage{fancyhdr}
\pagestyle{fancy}
\fancyhf{}
\rhead{\thepage}
\renewcommand{\headrulewidth}{0pt}

\begin{document}

% ====================
% PROJECT SUMMARY (1 page maximum)
% ====================

\section*{Project Summary}

\subsection*{Overview}
Provide a concise 1-2 paragraph description of the proposed research. This should be understandable to a scientifically literate reader who is not a specialist in your field.

\subsection*{Intellectual Merit}
Describe how the project advances knowledge within its field and across different fields. Address:
\begin{itemize}
    \item How the project advances understanding in the field
    \item Innovative aspects of the research
    \item Qualifications of the research team
    \item Adequacy of resources
\end{itemize}

\subsection*{Broader Impacts}
Describe the potential benefits to society and contributions to desired societal outcomes. Address one or more of the following:
\begin{itemize}
    \item Advancing discovery and understanding while promoting teaching and learning
    \item Broadening participation of underrepresented groups in STEM
    \item Disseminating broadly to enhance scientific and technological understanding
    \item Benefits to society (economic development, health, quality of life, national security, etc.)
    \item Developing the scientific workforce and enhancing research infrastructure
\end{itemize}

\newpage

% ====================
% PROJECT DESCRIPTION (15 pages maximum)
% ====================

\section*{Project Description}

\section{Introduction and Background}
\subsection{Current State of Knowledge}
Provide context for your proposed research. Review relevant literature and establish what is currently known in the field.

\subsection{Knowledge Gap}
Clearly identify the gap in current knowledge or understanding that your project will address. Explain why this gap is significant.

\subsection{Preliminary Work and Feasibility}
Describe any preliminary work that demonstrates the feasibility of your approach. Highlight your team's qualifications and prior accomplishments.

\section{Research Objectives and Hypotheses}
\subsection{Overall Goal}
State the overarching long-term goal of your research program.

\subsection{Specific Objectives}
List 2-4 specific, measurable objectives for this project:
\begin{enumerate}
    \item \textbf{Objective 1:} Clearly stated objective
    \item \textbf{Objective 2:} Second objective
    \item \textbf{Objective 3:} Third objective
\end{enumerate}

\subsection{Hypotheses}
State your testable hypotheses explicitly.

\section{Research Plan}
\subsection{Objective 1: [Title]}
\subsubsection{Rationale}
Explain why this objective is important and how it addresses the knowledge gap.

\subsubsection{Approach and Methods}
Describe in detail how you will accomplish this objective. Include:
\begin{itemize}
    \item Experimental design or computational approach
    \item Methods and procedures
    \item Data collection and analysis
    \item Controls and validation
\end{itemize}

\subsubsection{Expected Outcomes}
Describe what results you expect and how they will advance the field.

\subsubsection{Potential Challenges and Alternatives}
Identify potential obstacles and describe alternative approaches.

\subsection{Objective 2: [Title]}
[Repeat same structure as Objective 1]

\subsection{Objective 3: [Title]}
[Repeat same structure as Objective 1]

\section{Timeline and Milestones}
Provide a detailed timeline showing when each objective will be addressed:

\begin{center}
\begin{tabular}{|l|p{3cm}|p{3cm}|p{3cm}|}
\hline
\textbf{Activity} & \textbf{Year 1} & \textbf{Year 2} & \textbf{Year 3} \\
\hline
Objective 1 activities & Months 1-6: ... & & \\
\hline
Objective 2 activities & Months 7-12: ... & Months 13-18: ... & \\
\hline
Objective 3 activities & & Months 19-24: ... & Months 25-36: ... \\
\hline
Publications & & Submit paper 1 & Submit papers 2-3 \\
\hline
\end{tabular}
\end{center}

\section{Broader Impacts}
\textit{Note: Broader Impacts must be substantive, not perfunctory. Integrate throughout proposal.}

\subsection{Educational Activities}
Describe specific educational activities integrated with the research:
\begin{itemize}
    \item Curriculum development
    \item Training of graduate and undergraduate students
    \item K-12 outreach programs
    \item Public science communication
\end{itemize}

\subsection{Broadening Participation}
Describe concrete efforts to broaden participation of underrepresented groups:
\begin{itemize}
    \item Recruitment strategies
    \item Mentoring programs
    \item Partnerships with minority-serving institutions
    \item Measurable outcomes
\end{itemize}

\subsection{Dissemination and Outreach}
Describe plans for broad dissemination:
\begin{itemize}
    \item Open-access publications
    \item Data and code sharing (repositories, licenses)
    \item Conference presentations and workshops
    \item Public engagement activities
\end{itemize}

\subsection{Societal Benefits}
Explain potential benefits to society:
\begin{itemize}
    \item Economic development
    \item Health and quality of life improvements
    \item Environmental sustainability
    \item National security (if applicable)
\end{itemize}

\subsection{Assessment of Broader Impacts}
Describe how you will measure the success of broader impacts activities. Include specific, measurable outcomes.

\section{Results from Prior NSF Support}
\textit{Required if PI or co-PI has received NSF funding in the past 5 years}

\subsection{Award Title and Number}
Award Number: NSF-XXXXX, Amount: \$XXX,XXX, Period: MM/YY - MM/YY

\subsection{Intellectual Merit}
Summarize research accomplishments and findings from prior award.

\subsection{Broader Impacts}
Describe broader impacts activities and outcomes from prior award.

\subsection{Publications}
List publications resulting from prior NSF support (up to 5 most significant):
\begin{enumerate}
    \item Author, A.A., et al. (Year). Title. \textit{Journal}, vol(issue), pages.
\end{enumerate}

\newpage

% ====================
% REFERENCES CITED (No page limit)
% ====================

\section*{References Cited}

\begin{thebibliography}{99}

\bibitem{ref1}
Author, A.A., \& Author, B.B. (2023). Article title. \textit{Journal Name}, \textit{45}(3), 123-145.

\bibitem{ref2}
Author, C.C., Author, D.D., \& Author, E.E. (2022). Book title. Publisher.

\bibitem{ref3}
Author, F.F., et al. (2021). Another article. \textit{Nature}, \textit{590}, 234-238.

% Add more references as needed

\end{thebibliography}

\newpage

% ====================
% BUDGET JUSTIFICATION (3-5 pages typical)
% Note: Budget is submitted separately in NSF's systems
% This justifies the budget requests
% ====================

\section*{Budget Justification}

\subsection*{A. Senior Personnel}
\textbf{PI Name (X\% academic year, Y summer months):} Justify percent effort and role in project. Summer salary calculated as X/9 of academic year salary.

\textbf{Co-PI Name (X\% academic year):} Justify role and effort.

\subsection*{B. Other Personnel}
\textbf{Postdoctoral Researcher (1.0 FTE, Years 1-3):} Justify need for postdoc, qualifications required, and role in project. Salary: \$XX,XXX/year.

\textbf{Graduate Student (2 students, Years 1-3):} Justify need, training opportunities, and project contributions. Stipend: \$XX,XXX/year per student.

\textbf{Undergraduate Researchers (2 students/year):} Describe research training opportunities. Hourly wage: \$XX/hour.

\subsection*{C. Fringe Benefits}
List fringe benefit rates for each personnel category as determined by institution.

\subsection*{D. Equipment (\$5,000+)}
\textbf{Instrument Name (\$XX,XXX):} Justify need, explain why existing equipment inadequate, describe how it enables proposed research.

\subsection*{E. Travel}
\textbf{Domestic Conference Travel (\$X,XXX/year):} Justify conference attendance for dissemination (1-2 conferences/year for PI and students).

\textbf{Field Work Travel (\$X,XXX):} If applicable, justify field site visits.

\subsection*{F. Participant Support Costs}
\textit{If hosting workshop, summer program, etc.}

Stipends, travel, and per diem for XX participants attending [workshop/program name].

\subsection*{G. Other Direct Costs}
\textbf{Materials and Supplies (\$X,XXX/year):} Itemize major categories (e.g., chemicals, consumables, software licenses).

\textbf{Publication Costs (\$X,XXX):} Budget for open-access publication fees (estimate X papers @ \$X,XXX each).

\textbf{Subaward to Partner Institution (\$XX,XXX):} Justify collaboration and subaward amount.

\textbf{Other:} Justify any other costs.

\subsection*{H. Indirect Costs}
Calculated at XX\% of Modified Total Direct Costs (institution's negotiated rate).

\newpage

% ====================
% DATA MANAGEMENT PLAN (2 pages maximum)
% ====================

\section*{Data Management Plan}

\subsection*{Types of Data}
Describe the types of data to be generated by the project:
\begin{itemize}
    \item Experimental data (e.g., measurements, observations)
    \item Computational data (e.g., simulation results, models)
    \item Metadata describing data collection and processing
\end{itemize}

\subsection*{Data and Metadata Standards}
Describe standards to be used for data format and metadata:
\begin{itemize}
    \item File formats (e.g., HDF5, NetCDF, CSV)
    \item Metadata standards (e.g., Dublin Core, domain-specific standards)
    \item Documentation of data collection and processing
\end{itemize}

\subsection*{Policies for Access and Sharing}
Describe how data will be made accessible:
\begin{itemize}
    \item Repository for data deposition (e.g., Dryad, Zenodo, domain-specific archive)
    \item Timeline for public release (immediately upon publication, or within X months)
    \item Access restrictions (if any) and justification
    \item Embargo periods (if applicable)
\end{itemize}

\subsection*{Policies for Re-use, Redistribution}
Describe terms for re-use:
\begin{itemize}
    \item Licensing (e.g., CC0, CC-BY, specific data use agreement)
    \item Attribution requirements
    \item Restrictions on commercial use (if any)
\end{itemize}

\subsection*{Plans for Archiving and Preservation}
Describe long-term preservation strategy:
\begin{itemize}
    \item Repository selection (long-term, stable repositories)
    \item Preservation period (minimum 3-5 years post-project)
    \item Data formats for long-term preservation
    \item Institutional commitments
\end{itemize}

\subsection*{Roles and Responsibilities}
Identify who is responsible for data management implementation.

\end{document}

% ====================
% ADDITIONAL DOCUMENTS (submitted separately in NSF system)
% ====================

% 1. BIOGRAPHICAL SKETCH (3 pages per person)
%    - Use NSF-approved format (SciENcv or NSF template)
%    - Professional preparation
%    - Appointments
%    - Products (up to 5 most relevant, up to 5 other significant)
%    - Synergistic activities

% 2. CURRENT AND PENDING SUPPORT
%    - All current and pending support for all senior personnel
%    - Use NSF format
%    - Check for overlap with proposed project

% 3. FACILITIES, EQUIPMENT, AND OTHER RESOURCES
%    - Describe available facilities and equipment
%    - Computational resources
%    - Laboratory space
%    - Other resources supporting the project

% ====================
% FORMATTING CHECKLIST
% ====================

% ☐ Margins: 1 inch on all sides
% ☐ Font: Times Roman 11pt or larger (or equivalent)
% ☐ Line spacing: Single spacing acceptable
% ☐ Project Summary: 1 page, includes Overview, Intellectual Merit, Broader Impacts
% ☐ Project Description: 15 pages maximum
% ☐ References Cited: No page limit, consistent formatting
% ☐ Biographical Sketches: 3 pages per person, NSF-approved format
% ☐ Budget Justification: Detailed and reasonable
% ☐ Data Management Plan: 2 pages maximum
% ☐ Current & Pending: Complete and accurate
% ☐ Facilities: Adequate resources described
% ☐ Broader Impacts: Substantive and integrated throughout
% ☐ All required sections included

% ====================
% SUBMISSION NOTES
% ====================

% 1. Submit through Research.gov or Grants.gov
% 2. Follow your institution's internal deadlines (usually 3-5 days before NSF deadline)
% 3. Obtain institutional approval before submission
% 4. Ensure all senior personnel have NSF IDs
% 5. Budget prepared in NSF's system (separate from this document)
% 6. Check program-specific requirements (may differ from standard grant)
% 7. Contact Program Officer for guidance (encouraged but not required)

