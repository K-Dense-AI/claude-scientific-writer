% NIH Specific Aims Page Template
% THE MOST CRITICAL PAGE OF YOUR NIH PROPOSAL
% 1 page maximum - strictly enforced
% Last updated: 2024

\documentclass[11pt,letterpaper]{article}

% Formatting
\usepackage[margin=0.5in]{geometry}  % 0.5 inch minimum margins
\usepackage{helvet}  % Arial-like font
\renewcommand{\familydefault}{\sfdefault}

\usepackage{setspace}
\usepackage{color}
\usepackage{soul}  % For highlighting (remove in final version)

% Remove page numbers (optional)
\pagestyle{empty}

\begin{document}

% Optional: Highlight template text to remind yourself to replace
% Remove \hl{} and color in final version
\definecolor{highlight}{RGB}{255,255,200}
\sethlcolor{highlight}

% ====================
% SPECIFIC AIMS PAGE
% ====================

\begin{center}
\textbf{\large Your Project Title Here: Concise and Descriptive}
\end{center}

\vspace{0.3cm}

% OPENING PARAGRAPH: The Hook and Gap
% 2-3 sentences establishing significance and the knowledge gap

\textbf{[Disease/condition]} affects \textbf{[number]} people worldwide and results in \textbf{[burden: mortality, morbidity, cost]}. \textbf{[Current treatment/understanding]} has improved outcomes, but \textbf{[limitation/gap]} remains a critical barrier to \textbf{[desired outcome]}. Understanding \textbf{[specific mechanism/relationship]} is essential for \textbf{[future advance: therapy, prevention, diagnosis]}.

\vspace{0.2cm}

% LONG-TERM GOAL
% 1 sentence on your overarching research vision

Our \textbf{long-term goal} is to \textbf{[overarching vision: develop cure, understand mechanism, improve treatment]} for \textbf{[disease/population]}. 

\vspace{0.2cm}

% OBJECTIVE AND CENTRAL HYPOTHESIS
% 1-2 sentences on what THIS proposal will accomplish

The \textbf{objective} of this proposal is to \textbf{[specific objective for this project]}. Our \textbf{central hypothesis} is that \textbf{[clearly stated, testable hypothesis]}.

\vspace{0.2cm}

% RATIONALE
% 2-3 sentences explaining WHY you expect success (preliminary data!)

This hypothesis is based on our \textbf{preliminary data} showing that \textbf{[key preliminary finding 1]} and \textbf{[key preliminary finding 2]}. These findings suggest that \textbf{[mechanistic explanation or expected outcome]}.

\vspace{0.2cm}

% TRANSITION TO AIMS
% 1 sentence introducing the specific aims

To test this hypothesis and achieve our objective, we will pursue the following \textbf{Specific Aims}:

\vspace{0.3cm}

% ====================
% SPECIFIC AIM 1
% ====================

\noindent\textbf{Specific Aim 1: [Concise, active verb title describing what you'll do].}

\textit{Working Hypothesis:} \hl{State testable hypothesis for this aim.}

We will \textbf{[approach/method]} to determine \textbf{[what you'll learn]}. We will use \textbf{[model system/approach]} to test whether \textbf{[specific prediction]}. 

\textbf{Expected Outcome:} We expect to find that \textbf{[predicted result]}. This outcome will demonstrate that \textbf{[significance of finding]} and will be \textbf{[positive/negative/innovative/transformative]} because \textbf{[why it matters]}.

\vspace{0.3cm}

% ====================
% SPECIFIC AIM 2
% ====================

\noindent\textbf{Specific Aim 2: [Title of second aim].}

\textit{Working Hypothesis:} \hl{Testable hypothesis for Aim 2.}

Building on Aim 1, we will \textbf{[approach]} to \textbf{[objective]}. We will employ \textbf{[method/technique]} in \textbf{[model/population]} to test the hypothesis that \textbf{[specific prediction]}.

\textbf{Expected Outcome:} These studies will reveal \textbf{[predicted finding]}. This is significant because \textbf{[impact on field/understanding]}.

\vspace{0.3cm}

% ====================
% SPECIFIC AIM 3 (OPTIONAL)
% ====================

\noindent\textbf{Specific Aim 3: [Title of third aim].}

\textit{Working Hypothesis:} \hl{Testable hypothesis for Aim 3.}

To translate findings from Aims 1-2, we will \textbf{[approach]} to determine \textbf{[translational objective]}. We will \textbf{[method]} using \textbf{[clinically relevant model/patient samples]} to test whether \textbf{[translational prediction]}.

\textbf{Expected Outcome:} We anticipate that \textbf{[result]}, which will provide \textbf{[proof-of-concept/validation/mechanism]} for \textbf{[therapeutic/diagnostic/preventive strategy]}.

\vspace{0.3cm}

% ====================
% PAYOFF PARAGRAPH
% ====================

% 2-3 sentences on IMPACT, INNOVATION, and FUTURE DIRECTIONS

\textbf{Impact and Innovation:} This project is \textbf{innovative} because it \textbf{[novel aspect: new concept, method, approach, application]}. The proposed research is \textbf{significant} because it will \textbf{[advance the field by...]} and will ultimately lead to \textbf{[long-term impact: improved treatment, new therapeutic target, diagnostic tool]}. Upon completion of these studies, we will be positioned to \textbf{[next steps: clinical trial, mechanistic studies, therapeutic development]}.

\vspace{0.5cm}

% ====================
% ALTERNATIVE STRUCTURE (if preferred)
% ====================

% Some successful Specific Aims pages use this alternative structure:
% - Open with hook (same as above)
% - State long-term goal and objective (same)
% - Present central hypothesis with 2-3 supporting pieces of preliminary data
% - Then state: "We will test this hypothesis through three Specific Aims:"
% - List aims more concisely (1-2 sentences each, plus expected outcome)
% - Conclude with payoff paragraph emphasizing innovation, significance, impact

\end{document}

% ====================
% TIPS FOR WRITING SPECIFIC AIMS
% ====================

% 1. START WITH A HOOK
%    - Open with the big picture: disease burden, societal cost, mortality
%    - Use compelling statistics
%    - Make it clear why anyone should care

% 2. IDENTIFY THE GAP
%    - What's currently known?
%    - What's the critical barrier or unknown?
%    - Why does it matter?

% 3. STATE YOUR HYPOTHESIS EXPLICITLY
%    - Clear, testable hypothesis
%    - Not "We hypothesize that we will study..." (that's not a hypothesis!)
%    - "We hypothesize that [mechanism] causes [outcome]"

% 4. SHOW PRELIMINARY DATA
%    - Demonstrate feasibility
%    - Prove you're not starting from scratch
%    - Build confidence in your approach

% 5. THREE AIMS (TYPICALLY)
%    - Can be 2 or 4, but 3 is most common
%    - Aims should be related but somewhat independent
%    - Failure of one aim shouldn't sink the whole project
%    - Aims can build on each other (Aim 1 → Aim 2 → Aim 3)

% 6. EACH AIM SHOULD HAVE:
%    - Clear title (active verb)
%    - Working hypothesis
%    - Approach/method
%    - Expected outcome
%    - Significance/impact

% 7. END WITH PAYOFF
%    - Innovation: What's new/different?
%    - Significance: Why does it matter?
%    - Impact: What will change?
%    - Future: Where does this lead?

% 8. COMMON MISTAKES TO AVOID
%    - Too much background (this is not a mini-review)
%    - Vague hypotheses or objectives
%    - Missing expected outcomes
%    - No preliminary data mentioned
%    - Too ambitious (can't do it all in 5 years)
%    - Not addressing innovation and significance
%    - Poor logical flow between aims
%    - Exceeding 1 page (auto-reject!)

% 9. FORMATTING RULES (STRICTLY ENFORCED)
%    - 1 page maximum (including all text, no figures typically)
%    - Arial 11pt minimum (or equivalent)
%    - 0.5 inch margins minimum
%    - Any spacing (single, 1.5, double acceptable)
%    - No smaller fonts allowed (even for superscripts/subscripts)

% 10. REVISION STRATEGY
%    - Write, get feedback, revise 10+ times
%    - Every word must earn its place
%    - Test on non-specialist colleagues
%    - Read aloud to check flow
%    - Have it reviewed by successful R01 holders
%    - Mock study section review

% ====================
% EXAMPLES OF STRONG OPENING SENTENCES
% ====================

% DISEASE BURDEN APPROACH:
% "Alzheimer's disease (AD) affects 6.7 million Americans and will cost $345 billion in 2023, 
% yet no disease-modifying therapies exist."

% MECHANISTIC GAP APPROACH:
% "Despite decades of research, the molecular mechanisms driving metastasis remain poorly understood, 
% limiting our ability to develop effective therapies for the 90% of cancer deaths caused by metastatic disease."

% TRANSLATIONAL APPROACH:
% "Current immunotherapies fail in 70% of patients with melanoma, largely because we cannot predict 
% who will respond, highlighting an urgent need for biomarkers of treatment response."

% ====================
% REMEMBER
% ====================

% The Specific Aims page is often the ONLY page reviewers read carefully before 
% forming their initial opinion. A weak Specific Aims page can doom an otherwise 
% excellent proposal. Invest the time to make it compelling, clear, and concise.

% Get feedback from:
% - Successful R01 awardees in your field
% - Grant writing office at your institution
% - Colleagues who've served on NIH study sections
% - Non-specialists (if they can't understand it, reviewers may struggle too)

