% ==============================================================================
% Research Poster Template - tikzposter
% ==============================================================================
% A modern, colorful poster template using tikzposter
% Customize themes, colors, and content as needed
% ==============================================================================

\documentclass[
  25pt,                      % Font scaling
  a0paper,                   % Paper size
  portrait,                  % Orientation
  margin=10mm,               % Outer margins (minimal for full page)
  innermargin=15mm,          % Space inside blocks
  blockverticalspace=15mm,   % Space between blocks
  colspace=15mm,             % Space between columns
  subcolspace=8mm            % Space between subcolumns
]{tikzposter}

% Packages
\usepackage{graphicx}
\usepackage{amsmath,amssymb}
\usepackage{booktabs}
\usepackage{qrcode}
\usepackage{hyperref}

% Theme selection (uncomment your choice)
\usetheme{Rays}           % Modern with radiating background
% \usetheme{Wave}         % Clean with decorative wave
% \usetheme{Board}        % Board-style with texture
% \usetheme{Envelope}     % Minimal with envelope corners
% \usetheme{Default}      % Professional with lines

% Color style (uncomment your choice)
\usecolorstyle{Denmark}   % Professional blue
% \usecolorstyle{Australia} % Warm colors
% \usecolorstyle{Sweden}    % Cool tones
% \usecolorstyle{Britain}   % Earth tones

% Custom color scheme (optional - comment out if using built-in)
% \definecolorstyle{CustomStyle}{
%   \definecolor{colorOne}{RGB}{0,51,102}      % Dark blue
%   \definecolor{colorTwo}{RGB}{255,204,0}     % Gold
%   \definecolor{colorThree}{RGB}{204,0,0}     % Red
% }{
%   % Background Colors
%   \colorlet{backgroundcolor}{white}
%   \colorlet{framecolor}{colorOne}
%   % Title Colors
%   \colorlet{titlefgcolor}{white}
%   \colorlet{titlebgcolor}{colorOne}
%   % Block Colors
%   \colorlet{blocktitlebgcolor}{colorOne}
%   \colorlet{blocktitlefgcolor}{white}
%   \colorlet{blockbodybgcolor}{white}
%   \colorlet{blockbodyfgcolor}{black}
% }
% \usecolorstyle{CustomStyle}

% ==============================================================================
% POSTER CONTENT - CUSTOMIZE BELOW
% ==============================================================================

\title{Your Research Title: Concise and Descriptive}
\author{Author One\textsuperscript{1}, Author Two\textsuperscript{2}, \underline{Presenting Author}\textsuperscript{1}}
\institute{
  \textsuperscript{1}Department, University Name, City, Country\\
  \textsuperscript{2}Research Institute Name, City, Country
}

% Title matter (logos)
\titlegraphic{
  \includegraphics[width=0.1\textwidth]{logo1.pdf}
  \hspace{3cm}
  \includegraphics[width=0.1\textwidth]{logo2.pdf}
}

\begin{document}

\maketitle

% ==============================================================================
% MAIN CONTENT - 3 COLUMN LAYOUT
% ==============================================================================

\begin{columns}
  
  % ============================================================================
  % LEFT COLUMN
  % ============================================================================
  \column{0.33}
  
  \block{Introduction}{
    \textbf{Background}
    
    Brief context about your research area. One to two sentences establishing the importance of the topic.
    
    \vspace{0.5cm}
    
    \textbf{Problem Statement}
    
    What gap or challenge does your work address? Why is this important? One to two sentences.
    
    \vspace{0.5cm}
    
    \textbf{Research Objective}
    
    Clear, concise statement of what you set out to do in this study.
  }
  
  \block{Methods}{
    \textbf{Study Design}
    \begin{itemize}
      \item Experimental approach or study type
      \item Sample size: n = X participants/samples
      \item Key inclusion/exclusion criteria
    \end{itemize}
    
    \vspace{0.5cm}
    
    \textbf{Procedures}
    \begin{itemize}
      \item Main experimental steps
      \item Key measurements or interventions
      \item Data collection approach
    \end{itemize}
    
    \vspace{0.5cm}
    
    \textbf{Analysis}
    \begin{itemize}
      \item Statistical methods used
      \item Software/tools (e.g., R 4.3, Python)
      \item Significance threshold (p < 0.05)
    \end{itemize}
    
    \vspace{0.5cm}
    
    % Optional: Methods flowchart
    \begin{tikzfigure}
      \includegraphics[width=0.9\linewidth]{methods_diagram.pdf}
    \end{tikzfigure}
  }
  
  % ============================================================================
  % MIDDLE COLUMN
  % ============================================================================
  \column{0.33}
  
  \block{Results: Finding 1}{
    Brief description of your first main result. What did you observe?
    
    \begin{tikzfigure}
      \includegraphics[width=0.95\linewidth]{figure1.pdf}
    \end{tikzfigure}
    
    \textbf{Figure 1:} Descriptive caption explaining the figure. Include key statistics (Mean ± SD, n=X, **p<0.01).
  }
  
  \block{Results: Finding 2}{
    Brief description of your second main result.
    
    \begin{tikzfigure}
      \includegraphics[width=0.95\linewidth]{figure2.pdf}
    \end{tikzfigure}
    
    \textbf{Figure 2:} Another key result showing comparison, trend, or correlation.
  }
  
  % ============================================================================
  % RIGHT COLUMN
  % ============================================================================
  \column{0.33}
  
  \block{Results: Finding 3}{
    Brief description of your third main result or validation.
    
    \begin{tikzfigure}
      \includegraphics[width=0.95\linewidth]{figure3.pdf}
    \end{tikzfigure}
    
    \textbf{Figure 3:} Additional important finding or supporting data.
  }
  
  \block{Conclusions}{
    \textbf{Key Findings}
    \begin{itemize}
      \item \textbf{Main conclusion 1:} Impact and significance
      \item \textbf{Main conclusion 2:} Novel contribution
      \item \textbf{Main conclusion 3:} Practical implications
    \end{itemize}
    
    \vspace{0.5cm}
    
    \textbf{Limitations}
    \begin{itemize}
      \item Brief acknowledgment of study constraints
      \item Context for result interpretation
    \end{itemize}
    
    \vspace{0.5cm}
    
    \textbf{Future Directions}
    \begin{itemize}
      \item Ongoing or planned follow-up studies
      \item Broader applications of findings
    \end{itemize}
  }
  
  \block{Scan for More}{
    \begin{center}
      \qrcode[height=5cm]{https://doi.org/10.1234/your-paper}\\
      \vspace{0.5cm}
      \large Full paper, code, and data
    \end{center}
  }
  
\end{columns}

% ==============================================================================
% FOOTER (Full Width)
% ==============================================================================

\block[width=1.0\linewidth]{}{
  \footnotesize
  \begin{minipage}{0.7\textwidth}
    \textbf{References}
    \begin{enumerate}
      \item Author A et al. (2023). Title of paper. \textit{Journal Name}, 10(2), 123-145. doi:10.xxxx/xxxxx
      \item Author B et al. (2024). Title of paper. \textit{Conference Proceedings}.
      \item Author C et al. (2022). Title of paper. \textit{Journal Name}, 15(3), 456-478.
    \end{enumerate}
    
    \vspace{0.3cm}
    
    \textbf{Acknowledgments:} This work was supported by Funding Agency (Grant \#12345). We thank collaborators at Institution X and the Core Facility for technical support.
    
    \vspace{0.3cm}
    
    \textbf{Contact:} presenter.email@university.edu | Twitter: @labname | Website: labname.university.edu
  \end{minipage}%
  \hfill
  \begin{minipage}{0.25\textwidth}
    \raggedleft
    Conference Name 2024\\
    Location, Dates\\
    Poster \#XXX
  \end{minipage}
}

\end{document}

